\chapter{Lessons for C++ linear algebra library standardization}

When regular WG21 participants began discussing standardization of a
linear algebra library, many features, issues, and terms came up that
had appeared in linear algebra libraries many years before.  Some
participants lacked familiarity with lessons learned from older
projects, for example about separation of concerns between performance
and accuracy.

To avoid confusion and aid learning from history, we wrote this survey
of lessons learned from linear algebra libraries and standardization
efforts, in C++ and other languages (especially Fortran).  The latter
focuses on the BLAS \cite{BLAS-standard}, LINPACK
\cite{dongarra1979linpack}, and LAPACK \cite{LAPACK-Users-Guide}
libraries, with their long and successful history and frequent use in
C++ linear algebra libraries and applications.  This survey also puts
early C++ linear algebra libraries into a larger historical context,
to help expose motivations for design decisions.

This document does \emph{not} attempt a comprehensive survey of C++
linear algebra libraries.  Furthermore, the authors freely admit that
their background in scientific and engineering computation will
introduce bias.  The point is not to claim objectivity or
completeness, but rather to tell some stories that could help guide
the development of a standard C++ linear algebra library.  To this
end, one of the authors, Mark Hoemmen, will include anecdotes from his
experiences working in the Trilinos C++ project
\cite{heroux2005trilinos} since 2010.  The authors also include oral
histories that they collected circa 2018 from a POOMA project
developer, Chris Luchini \cite{hoemmen2018history}, as well as an
interview of Jack Dongarra taken for the Society of Industrial and
Applied Mathematics in 2005 \cite{dongarra2005history}.  Oral
histories also carry subjectivity, but nonetheless can help readers
understand design motivations.

We begin with Section \ref{S:90s}, an overview of 1990's trends in
computer hardware and applications that shaped early C++ linear
algebra libraries.  Section \ref{S:features} highlights features of
C++ linear algebra libraries, both historical and current, that we
think should guide development of a standard library.  Section
\ref{S:other-langs} surveys lessons learned from the BLAS, LINPACK,
and LAPACK.  Finally, Section \ref{S:other-standards} talks about
other standardization efforts that extend the BLAS.

\section{1990's trends leading to C++ linear algebra libraries}
\label{S:90s}

Two key features of modern C++ -- templates, and the Standard Template
Library -- both evolved in the 1990's.  That time also spawned other
trends concurrent to and possibly shaping the development of early C++
linear algebra libraries.  Surveying these trends can help us see why
we might want such libraries in the first place, and what motivated
the libraries' design decisions.  

We begin with object-oriented numerics in Section \ref{SS:90s:OON}.
Section \ref{SS:90s:demo} talks about the democratization of
high-performance computing, and Section \ref{SS:90s:multiphysics}
about increasing complexity in numerical simulations.  We finish with
the democratization of computer graphics in Section
\ref{SS:90s:graphics}.

\subsection{Object-Oriented Numerics}
\label{SS:90s:OON}

Summary: C++ library developers eagerly applied object-oriented
programming to numerical computation, including numerical linear
algebra.  This made libraries easier to use and more reusable.
Developers learned to use expression templates to avoid performance
issues of na\"ive encapsulation.  They also introduced a new interface
design antipattern by failing to factor classes by responsibility.

Linear algebra libraries in C++ at first evolved from the general
category of ``object-oriented programming'' (OOP).  One can see the
explosion of interest in so-called ``object-oriented numerics'' by the
large variety of conferences that sprang up in the
mid-1990's.\footnote{See e.g.,
  \url{http://www.math.unipd.it/~michela/OP.htm#conferences}.}  For
example, the First annual Object-Oriented Numerics Conference
(OON-SKI) took place in 1993.\footnote{See
  \url{https://www.hindawi.com/journals/sp/si/250702/}.} Rogue Wave, a
C++ compiler company, sponsored OON-SKI, so the conference logically
focuses on work in C++.  ``Numerics'' here means ``numerical
computation'' or ``scientific computing'': computations on large
arrays of floating-point numbers, doing things like simulating the
physical world or analyzing large quantities of data statistically.

Numerical library developers of the time saw OOP as a way to ``keep
the CPU-time constant while reducing the human time
drastically'' \cite{arge1996oon}.  OOP's encourages modularity and
encapsulation, and thus simplifies testing and reuse of software
components. It lets developers create ``flexible applications that can
be easily extended\dots.  In short, OOP encourages computer
implementations of mathematical abstractions'' \cite{arge1996oon}.

\subsubsection{Early lessons learned}

Summary: in the early 1990's, C++ developers working on
object-oriented numerics had already learned the following lessons:
\begin{itemize}
\item Libraries want users to write linear algebra expressions in code
  that look as much like mathematical notation as possible.
\item C++ linear algebra libraries can encapsulate some of the details
  of getting good performance, including parallel computation.
\item Na\"ive abstraction can hurt performance.
\item C++ expression templates can help improve performance while
  enabling interfaces that look like mathematical notation.
\end{itemize}

The introduction to the journal special issue for OON-SKI 1993
described the phenomenon of ``object-oriented numerics'' as follows:
\begin{quote}
  \ldots [W]e are observing the emergence of a subdiscipline: the use
  of object-oriented techniques in numerics. But what we are really
  seeing is something even more profound: finally rejoining of
  scientific computing with the science of computers. Traditionally,
  programming has been done by engineers, physicists and
  mathematicians with little or no training in computer science. Now,
  however, we are seeing an infusion of ideas coming from [the]
  computer science world into the scientific computing world, bringing
  along modern ideas on how to structure complex numerical
  code. Object-oriented techniques is [sic] merely one of many such
  ideas \cite{Vermeulen1993}.
\end{quote}
The introduction talks about issues like needing to teach compilers
how to fuse loops and avoid temporaries when doing overloaded-operator
arithmetic on arrays in C++. It also shows the existence of C++
libraries for a variety of applications, including a library of
multidimensional arrays, and the integration of C++ with
distributed-memory parallel computation.  This shows that C++
developers working on object-oriented numerics had already learned the
lessons mentioned in the summary above.

\subsubsection{Antipattern: Classes with too many responsibilities}

Summary: Some early object-oriented numerics libraries have objects
that can be in many different states, and many mutating instance
methods with state-dependent preconditions.  This is an antipattern
that adds complexity for both users and developers.  Application of
the One Responsibility Rule and Idiomatic use of the modern C++ type
system can correct this without sacrificing performance.

It's easy to take the phrase ``object-oriented programming'' for
granted, as equivalent to ``modern'' or ``not old fashioned.''
Proponents of object-oriented numerics contrasted their interfaces
with old-fashioned Fortran code: routines with cryptically short names
that take dozens of arguments and have thousands of lines of code, and
single objects with state scattered over many separate
variables \cite{arge1996oon}.  Object-oriented interfaces
grouped together state into nouns (e.g., ``matrix,'' ``vector'') and
verbs (e.g., ``dot product,'' ``matrix-vector multiply'') that users
found self-documenting.

This is overall a good thing.  For example, contrast ScaLAPACK's
parallel dense LU factorization interface, with its long list of
mysterious arguments and complicated parallel distribution setup
boilerplate, with the clean interfaces to the same functionality in
the Elemental \cite{poulson2013elemental} or Global Arrays
\cite{GlobalArrays} libraries.  However, a common design antipattern
in early object-oriented numerics libraries, is to let an object of a
single type have many possible states, with many mutating instance
methods whose preconditions depend on the object's state.\footnote{See
  e.g., \cite[p. 6]{arge1996oon}.  It's a bit unfair to pick on this
  one source, as the authors can think of many examples, including
  codes on which the authors work.}

This approach seems natural for many scientific and engineering
applications, that have a small number of ``objects'' representing
large memory allocations that the application modifies in place.
However, we consider this an ``antipattern'' for three reasons:
\begin{enumerate}
\item The resulting classes violate the One Responsibility
  Rule.\footnote{See e.g.,
    \url{http://wiki.c2.com/?OneResponsibilityRule}.}
\item Users and library developers both find it hard to keep in mind
  what operations are valid on an object at what times.
\item This approach fails to use C++'s type system and language
  features to help make the current state obvious to users.
\end{enumerate}
What we mean by the third point above, is that a library could
represent state transitions as \emph{type} transitions, that consume
an object of the old type to produce an object of the new type.
``Consuming'' an object could happen via move construction.  C++ has
``smart pointers'' that the library could use to let ownership of
large memory allocations pass from the old object to the new one,
without need for reallocation and copying.

As an example of this antipattern, consider sparse matrices
(\texttt{Epetra\_CrsMatrix}) in the Trilinos project's Epetra linear
algebra library \cite{heroux2005trilinos}.  One creates an Epetra
sparse matrix by giving its constructor an object representing the
parallel distribution of its rows.  At that point, the matrix is
literally empty.  One then calls mutating methods on the matrix to
insert sparse matrix entries, one or more at a time.  After one is
done inserting, one must call the matrix's \texttt{FillComplete}
method in order to prepare the matrix for linear solves.  The matrix's
type stays the same through all of these mutations.  Its type tells
one nothing about what operations are currently legal on the object.
For example, one can only change the matrix's sparsity structure
\emph{before} calling \texttt{FillComplete}, and one can only solve
linear systems with the matrix \emph{after} \texttt{FillComplete}.
The matrix class has several different constructors besides the one
mentioned here, and the set of valid pre-\texttt{FillComplete}
operations on the matrix (e.g., whether one is allowed to change its
sparsity structure, or whether one needs to use local or global column
indices when changing entries) depends entirely on which constructor
one uses.  Changing anything inside this class is risky, since most
unit tests and code exercise only common cases that do not cover all
states.  It would have been better for Epetra to have had separate
classes to represent various kinds of ``half-baked matrix'' vs.\ a
``fully baked'' matrix.  

Note that even though the Sparse BLAS interface standard gives
matrices to users by integer handle rather than by instance of a C++
class, it shares the same property that the handle does not change
after the equivalent of \texttt{FillComplete}
\cite[pp. 129--30]{BLAS-standard}.

\subsection{Democratization of high-performance computing}
\label{SS:90s:demo}

A phenomenon we call the ``democratization of high-performance
computing'' may have also contributed to the development of C++ linear
algebra libraries.  This phenomenon has three parts.  First,
high-performance, lower-cost workstations let users solve more
problems without resorting to expensive ``big iron.''\footnote{This is
  part of the ``killer micros'' revolution.  Killer micros were
  consumer-grade, low-cost microprocessors whose performance
  threatened and eventually overtook that of expensive vector
  supercomputers \cite{killermicros1991}.} Workstation hardware tracked
the 18-month Moore's Law performance curve, so users could just wait
for the next processor generation, instead of expensively optimizing
their code.  This gave developers more freedom to write more
complicated codes.  In turn, this moved them towards programming
languages like C++, that aid encapsulation and larger-scale software
architecture.\footnote{Fortran 90 has modules, but no ``instances of a
  class.''  Fortran 2003 has features more like C++ instance methods
  and polymorphism.  Note, however, that Fortran compilers tend to lag
  behind the latest Fortran standard, much more than C++ compilers lag
  behind the latest C++ standard.  The authors' general experience
  with relying on Fortran 2003 features has been poor.  For example,
  the use of these features in the ForTrilinos project proved
  unsustainable, and drove a shift back to assuming a modest subset of
  Fortran 90 on the Fortran side of this interface to the C++ linear
  algebra library Trilinos.}

Second, distributed-memory parallel computers emerged.  These new
systems did not rely on custom vectorizing Fortran compilers, as did
the older generation of expensive vector supercomputers.  Some of
these systems did not even have a Fortran compiler, or strongly
favored other programming languages.\footnote{For example, the
  technical report describing Connection Machine's CM-1 only mentions
  one programming language: *Lisp, an ANSI Common Lisp derivative
  \cite{kahle1989cm1}.  Applications running on CM-2 also used C*, a
  parallel superset of ANSI C.  CM-2 only later added CM Fortran, an
  extension of Fortran, in 1991 \cite[p. 7]{Kennedy2007}.  Other
  distributed-memory parallel computers of the era favored C and
  offered C++ compilers.  Library-oriented approaches to parallel
  computing, like PVM and MPI, came with C as well as Fortran
  bindings.}  New systems had new programming models anyway, which
encouraged writing new libraries in languages like C++.

Third, distributed-memory parallel computation and standard
programming models made it cheaper and easier to build
high-performance computers with large memory capacities.  The rapid
increase in workstation performance meant that users often only
reached for ``supercomputers'' when they need to solve problems too
big to fit on a single workstation.  Techniques like Network of
Workstations let users assemble a parallel computer with a large
memory, out of the large number of inexpensive workstations they
already had \cite{anderson1995case}.  Standard parallel programming
models like MPI\footnote{The Message Passing Interface, whose Version
  1.0 came out in 1994.} and OpenMP\footnote{A standard for
  directives-based shared-memory parallel programming, whose C and C++
  interface came out in 1997.} made Fortran less important and
simplified writing standard libraries.  The emerging World Wide Web in
turn accelerated such libraries' promulgation.

\subsection{Increasing complexity in numerical simulations}
\label{SS:90s:multiphysics}

Concurrent with the above trends was an increasing demand for more
complex numerical simulations.  The United States Department of
Energy's Accelerated Strategic Computing Initiative (ASCI) program,
founded in 1996 \cite{messina1996asci}, is one example of a program
that helped spur this demand.  An important part of ASCI was
``[s]oftware design and development'' for complicated ``multi-physics
scientific applications.''  Before ASCI, experience with such software
development was ``limited to a few isolated projects.''  ASCI aimed to
``carry out multiple substantial simulation projects that will provide
proofs of principle, at scale, of software development methodologies
and frameworks that provide portability, extensibility, modularity,
and maintainability while delivering acceptable performance''
\cite{messina1996asci}.

\subsection{Democratization of computer graphics}
\label{SS:90s:graphics}

The 1990's also saw a standardization of programming models for
computer graphics.  For example, OpenGL 1.0 came out in 1992
\cite{OpenGL-history}, and DirectX not long after.\footnote{See, e.g.,
  the following overview of the history of OpenGL and Direct3D:
  \url{https://news.ycombinator.com/item?id=2711231}.} Along with this
came new hardware and instruction sets, like MMX \cite{mittal1997mmx}
and 3DNow! \cite{oberman1999amd}, for accelerating graphics operations
on consumer processors.  Computer graphics depends heavily on
performing many small dense linear algebra operations.  This is a use
case that historically did not cross over much with non-graphics
scientific computation.  However, the scientific computing community
is starting to show more interest in so-called ``batched'' linear
algebra interfaces for performing many small dense operations in
parallel \cite{dongarra2016batched}.

\subsection{Summary of 1990's trends}
\label{SS:90s:summary}

A massive change in high-performance computer architectures and
programming models led software developers to start using C++ for
linear algebra libraries.  Concurrently, increasing interest in
computer graphics, and the availability of dedicated instruction sets
for accelerating primitive operations needed for graphics, stirred a
growing interest in linear algebra outside of traditional scientific
and engineering computing circles.  C++ linear algebra library
developers discovered that na\"ive encapsulation could have a
performance cost, and that C++ techniques like expression templates
could help reduce or eliminate this cost.

\section{Features of C++ linear algebra libraries}
\label{S:features}

In this section, we highlight common features of early C++ linear
algebra libraries, that we think express good lessons learned for any
C++ linear algebra library standardization effort.

\subsection{Templates to improve performance}
\label{SS:features:templates}

In summary:
\begin{itemize}
\item C++ code had a poor reputation for performance, compared with
  optimized C or Fortran code.
\item C++ templates, including expression templates, helped close the
  ``performance gap'' with optimized Fortran code.
\item C++ compilers of the 1990's did not necessarily have complete or
  correct implementations of templates.  Some linear algebra libraries
  excluded templates as a result.
\item Compilers are better at compiling templates now, so we don't
  have to be afraid to use them.  However, we need to respect concerns
  about compilation time.
\item With expression templates, users must be careful using
  \texttt{auto} for the left-hand side of an expression assignment.
\end{itemize}

C++ had a reputation for poor performance compared with Fortran. Even
developers willing to write C++ in ``numerical'' codes considered it
better to use C++ as a high-level coordination language, and reserve
lower-level languages like C or Fortran for tight loops.\footnote{See
  \cite{Arge1997}.  The Epetra linear algebra library in the Trilinos
  project \cite{heroux2005trilinos} has optional Fortran
  implementations of sparse matrix times multiple dense vectors at a
  time, since it found them faster than C++.  In 2010--11, one of the
  authors found Fortran versions of dense QR factorization kernels
  faster than equivalent C++ versions \cite{hoemmen2011comm}.}
Developers saw C++ templates, in particular expression templates, as
an optimization technique that could close the performance
gap. Expression templates would let developers write compact, abstract
code that ``looks like math,'' yet optimizes by fusing loops and
avoiding temporaries. For example, the Dr.\ Dobbs article
\cite{dobbsblitz1997} on the Blitz++ library \cite{blitz2005}, written
by the library's author, focuses on expression templates for vector
operations.

Developers also recognized the cost of virtual method calls in C++,
especially in inner tight loops, and used templates to reduce the cost
of run-time polymorphism.  For example, the Bernoulli Generic Matrix
Library uses the ``Barton-Nackman trick'' \cite{Barton1994},
a special case of the ``Curiously Recurring Template Pattern,'' to
turn run-time polymorphism into compile-time polymorphism.\footnote{In
  \cite{Mateev2000}, authors cite \cite{Veldhuizen2000}.} This pattern
shows up in other C++ linear algebra libraries; for example, Eigen
uses it for expression base classes.\footnote{See
  \url{http://eigen.tuxfamily.org/dox/TopicClassHierarchy.html}.}

Early libraries that relied on templates suffered due to incomplete
compiler implementations. For example, Blitz++'s installation process
exercises the compiler to test language feature compliance. Its User's
Guide recommends that if the compiler ``doesn't have member templates
and enum computations, just give up.''\cite[Section 1.4.3]{blitz2005}
A comparable library, POOMA (Parallel Object-Oriented Methods and
Applications),\footnote{See
  \url{http://www.nongnu.org/freepooma/tutorial/introduction.html}.}
pushed the boundaries of what the available C++ compilers could
handle. Chris Luchini, a POOMA developer, recalls that the project
exposed many compiler bugs \cite{hoemmen2018history}.  Many compilers
lagged behind the C++ standard, only implemented a subset of features,
and generated slow code \cite{Mateev2000}.

Software for scientific computing may need to build with several
different compilers and run on different kinds of hardware. Lack of
consistently complete implementations of templates challenged
portability requirements and restricted adoption. For example, in the
Trilinos software project, a requirement to support a C++ compiler
with incomplete template support drove the project to forbid templates
in its foundational linear algebra library, Epetra
\cite{hoemmen2018history}.

One of the authors, Mark Hoemmen, has had experience reducing compile
times and compiler memory usage of the Trilinos
\cite{heroux2005trilinos} and Kokkos\footnote{See
  \url{https://github.com/kokkos/kokkos}.} projects.  He has put
months and months of developer time into efforts like the following:
\begin{itemize}
\item explicit template instantiation;
\item splitting explicit instantiations into separate files, sometimes
  automatically, through build system scripts; and
\item reducing the number of separate instantiations of a templated
  class that add no value, because they generate identical code.
\end{itemize}
Indiscriminate use of templates can increase compile times, both by
preventing implementation hiding, and by proliferating instantiations.
Compilers sometimes generate slower code when a compilation unit is
too complicated or takes too long to build, since compilers want to
bound the time and memory they use.  However, neither of these
discourage most developers these days from using templated functions
and classes in the C++ Standard Template Library.  We think it is
possible and good to use templates, but developers need to think about
compilation time.  Users must also take responsibility for this, by
designing interfaces to hide implementations as much as possible, and
breaking up long compilation units.

Expression templates may hinder use of \texttt{auto} for the left-hand
side of expressions, and users need to be careful returning
expressions from a function without first assigning them to some
concrete linear algebra type.  This is because the type to which the
expression evaluates, is not necessarily a concrete matrix or vector
or other linear algebra object.  It may be some expression type, that
may hold references to concrete linear algebra objects.\footnote{See
  e.g., \url{http://eigen.tuxfamily.org/dox/TopicPitfalls.html}.}
Returning the expression may result in dangling references, if the
concrete linear algebra objects to which the expression refers have
fallen out of scope.  C++ copy elision\footnote{See
  \url{https://en.cppreference.com/w/cpp/language/copy_elision}.}
makes it impossible for a library to detect and report user code that
creates expressions on the left-hand side, such as \texttt{auto z = x
  + y}.
% This is because such code does not invoke the expression class'
% \texttt{operator=} method.

\subsection{Generic iteration over sequences}
\label{SS:features:iteration}

Summary: Developers discovered that C++ let them express generic
iteration over sequences in a high-level way, and generate optimized
code from the high-level specification.  A linear algebra or tensor
library may want to expose iteration over multidimensional sequences
or index spaces.  There are different ways of expressing this.

As experience with C++ templates increased, some developers applied
them to more radical code optimizations.  For example, the Bernoulli
Generic Matrix Library used templates to generate optimized sparse
matrix codes from a high-level specification \cite{Ahmed2000}.
Bernoulli used a kind of relational algebra (described in detail in
the PhD dissertation) that gives users a generic way to describe
operations over sequences, while optimizing by avoiding storage of
temporary intermediate sequences.  One can think of this as a
generalization of C++ iterators, and as a precursor to the C++ Ranges
proposal \cite{Niebler2018}.

By the time one of the authors (Mark Hoemmen) encountered the
Bernoulli project in the early 2000's, it had abandoned C++ templates
in favor of a code generation framework based on OCaml (or some other
ML derivative).  Our guess is that using OCaml to generate C code,
instead of using C++ templates, avoided C++ compiler correctness and
performance issues that were more common in the early 2000's.
Nevertheless, this suggests a lesson learned: the more ambitiously
complex a C++ library, the more programmer effort and expertise it
requires, and the bigger the risk for good performance on many
platforms.

The Kokkos C++ library\footnote{See
  \url{https://github.com/kokkos/kokkos}.} uses a different approach
for parallel iteration over multidimensional index ranges.  Users
specify the index ranges.  They may also specify nondefault features
as template parameters, like the computer architecture, the iteration
order, and whether to use tiling with compile-time sizes.  Kokkos uses
this specification to generate parallel code.  An important lesson
learned from Kokkos is that users need control over multidimensional
iteration order in order to get best performance.

Other approaches to multidimensional sequence iteration include
Einstein notation for tensor summation.
%
% TODO cite existing work on this
% 

\subsection{Engines: Implementation polymorphism}
\label{SS:features:engine}

Summary: C++ template partial specialization lets library developers
write linear algebra interfaces that are polymorphic on their
implementations.  This is an extension point, like the
\texttt{Allocator} template parameter in C++ Standard Library objects.
Library authors can and have used this for polymorphism on things like
the parallelization mechanism, the allocator, and the storage layout.

The POOMA (Parallel Object-Oriented Methods and Applications) project
was most active 1998-2000. POOMA's goal was to support structured grid
and dense array computations. As per oral history
\cite{hoemmen2018history} and POOMA's documentation\footnote{See
  \url{http://www.nongnu.org/freepooma/tutorial}.}, the team had a
particular interest in SGI Origin shared-memory parallel
computers. POOMA shares features with more recent linear algebra
libraries, such as polymorphism on storage layout and parallel
programming model, so it is worth studying for historical lessons.

POOMA's main data structure is Array, a multidimensional array. Array
has three template parameters: the rank (the number of dimensions),
the entry type (e.g., \texttt{double}), and the ``Engine.''  Engines
express where and how data are stored.  They implement access to
entries of the Array.  For example, depending on the POOMA Engine
used, Array entries could either actually exist in some storage
somewhere, or they could be computed on the fly at access time from
their indices.  POOMA Engines (in particular, the MultiPatch engine)
could also describe distributed-memory parallel data distribution.

This feature of implementation polymorphism by template specialization
shows up in many other libraries.  For example, the \texttt{mdspan}
multidimensional array proposal \cite{P0009r8} has an Accessor policy,
an optional template parameter that implements access to the entries
of the \texttt{mdspan}.  The Kokkos\footnote{See
  \url{https://github.com/kokkos/kokkos}.} library's \texttt{View}
multidimensional array class has ``execution space'' and ``memory
space'' template parameters, that control where data live (allocation
and deallocation) and how operations execute in parallel over those
data.

\subsection{Deducing return type of linear algebra expressions}
\label{SS:features:return-type-deduction}

Summary: Figuring out the right return type of a linear algebra
expression is a nonobvious design decision, that early C++ libraries
have already encountered.

One recent discussion point that has proven controversial, is how to
deduce the return type of an arithmetic expression involving linear
algebra objects with mixed element types.  For example, suppose that I
have a matrix \texttt{A} with elements of type
\texttt{complex<float>}, and I write the expression \texttt{4.2 * A}.
Should I get a matrix with elements of type \texttt{complex<double>},
since that avoids loss of accuracy when multiplying the
\texttt{double} $4.2$ by \texttt{complex<float>}?  What if I really
want \texttt{complex<float>}, for its lower storage requirements or
for compatibility with other interfaces?  It's easy to write
mixed-type expressions like \texttt{4.2 * A}, but not necessarily easy
to decide the type to which they should convert.

This is a simple example of a general problem: allowing arithmetic
expressions of linear algebra objects of mixed types means that a
library must decide the return type of such expressions.  The
aforementioned POOMA project encountered this general issue.  For
example, they learned the hard way that C++ does not permit
``templating on return type.''\footnote{See
  \url{http://www.nongnu.org/freepooma/tutorial/tut-03.html}.}

\subsection{Lazy evaluation may not always pay}
\label{SS:features:lazy}

Summary: Implementations of expression templates for linear algebra
expressions may lose performance and even get the wrong answer, if
they evaluate all expressions lazily.  Fixing this in the library
turns library developers into compiler authors and introduces a
complexity / performance trade-off.

C++ expression templates are one way to implement \emph{lazy
  evaluation} of arithmetic expressions involving linear algebra
objects.  Natural implementations of expression templates lead to
fully lazy evaluation: No actual computation happens until assignment.
However, multiple C++ linear algebra libraries have learned the lesson
that fully lazy evaluation does not always pay.  For example, for
algorithms with reuse, like dense matrix-matrix multiply, lazy
evaluation can give incorrect results for expressions where the output
alias the input (like \texttt{A = A * A} where \texttt{A} is a
matrix).  This is why the Eigen library's expressions have an option
to turn off lazy evaluation, and do so by default for some kinds of
expressions.  Furthermore, allocating a temporary result and/or eager
evaluation of subexpressions may be faster in some cases.\footnote{See
  \url{http://eigen.tuxfamily.org/dox/TopicLazyEvaluation.html}.}
This is true not necessarily just for expressions whose computations
have significant reuse, like matrix-matrix multiply, but also for some
expressions that ``stream'' through the entries of vectors or
matrices.  For example, fusing too many loops may thrash the cache or
cause register spilling, so deciding whether to evaluate eagerly or
lazily may require hardware-specific
information \cite{siek2008build}. It's possible to encode
many such compilation decisions in a pure C++ library with
architecture-specific parameters.\footnote{See, e.g., the following
  2013 Eigen presentation:
  \url{http://downloads.tuxfamily.org/eigen/eigen_CGLibs_Giugno_Pisa_2013.pdf}.}
However, this burdens the library with higher implementation
complexity and increased compilation time.  Library designers
may prefer a simpler interface that excludes expressions with reuse
(that have correctness issues with lazy evaluation) and lets users
decide where temporaries get allocated.

\subsection{Vector spaces and parallel data distributions}
\label{SS:features:spaces}

Summary: A ``vector space'' is a useful generalization of the number
of rows or columns of a matrix.  It can pay to expose this
generalization to library users, though a tensor library may find this
abstraction less useful.

Linear algebra libraries all must figure out what to do if users
attempt to perform operations on objects with incompatible dimensions.
For example, it does not make sense to add together two vectors $x$
and $y$ with different lengths, or to multiply two dense matrices $A$
and $B$ if $A$'s number of columns does not equal $B$'s number of
rows.  The only complication is whether the library has this
information at compile time, or must check at run time.

This gets more complicated, though, if we generalize ``compatible
dimensions'' to the mathematical idea of a ``vector space.''  Two
vectors might have the same dimensions, but it still might not make
sense to add them together.  For example, I can't add a coordinate in
3-D Euclidean space to a quadratic polynomial with real coefficients,
just like I can't add meters to seconds.  Two vectors from different
vector spaces may not be addable together, even though the two spaces
have the same dimension or are otherwise isomorphic.  Like a physical
unit, a vector space is a kind of ``metadata.''  Equating all
isomorphic vector spaces strips off their metadata.

The ``vector space as metadata'' idea also can apply to parallel data
distributions.  We use the term \emph{parallel data distribution} to
refer either to a particular distribution of data over a
distributed-memory parallel computer, or to the memory affinity of
data on shared-memory parallel computers with nonuniform memory access
(NUMA).  It may even be mathematically correct to add two vectors from
the same (mathematical) vector space but with different parallel
distributions.  However, doing so would require communication.
``Communication'' here may mean different things on different parallel
computers, but generally implies some combination of moving data
between processors (either through a memory hierarchy, or across a
network), and synchronization between processors.  Communication is
expensive relative to floating-point arithmetic
\cite{wulf1995hitting,Blackford1997}.  It also affects correctness,
because it may introduce deadlock, depending on what surrounding code
does.  Thus, programmers like to see communication made explicit, even
if it is hidden behind a convenient interface.

A natural way for a linear algebra library to expose parallel data
distributions to users is by folding the distribution into a vector
space.  For example, the library's vector space would not only give
the ``global'' dimension of a vector, but would express which vector
entries ``live'' on which parallel processes / in which NUMA memory
affinity regions.  Linear algebra libraries in the Trilinos C++
software project \cite{heroux2005trilinos} do this (e.g., in Epetra
and Tpetra, a vector space / distribution is called a ``Map''), and
they give users a way to query vector spaces for ``sameness.''

One issue with this approach is how to generalize it in a usable way
to tensors.  Tensors have more spaces than matrices.  Users may find
it hard to keep track of which matter for the operations that they
want to perform.

\section{Lessons learned from Fortran libraries}
\label{S:other-langs}

We begin with Section \ref{SS:other-langs:LINPACK}.  This uses Jack
Dongarra's oral history of the LINPACK project and examples from other
libraries to show that one of the most important features of a linear
algebra library, is that it is generic on the matrix element type.  In
Section \ref{SS:other-langs:BLAS}, the example of the BLAS illustrates
that it's helpful to separate libraries or levels of abstraction, by
whether their developers are more likely to be numerical linear
algebra experts or performance tuning experts.  Finally, the example
of High Performance Fortran in Section \ref{SS:other-langs:HPF}
teaches that an interface should make expensive operations explicit,
whenever possible.

% , and that good library interface abstractions can accomplish
% most if not all of what a language can do.

\subsection{LINPACK: One can write algorithms
  generic on the matrix element type}
\label{SS:other-langs:LINPACK}

Summary: The LINPACK project shows that it's possible to write linear
algebra algorithms that are generic on the matrix element type, even
in languages (like Fortran 77) that lack polymorphism.  Genericity is
one of the most important features that a C++ linear algebra library
can offer.

Dongarra gives an oral history \cite{dongarra2005history} of
standardization of popular Fortran linear algebra libraries, including
EISPACK and LINPACK.  LINPACK \cite{dongarra1979linpack} is a library
for solving dense linear systems, and EISPACK \cite{garbow1974eispack}
is for solving dense eigenvalue problems.  The two libraries together
are precursors to LAPACK \cite{anderson1990lapack}, that combines
functionality from both.  Dongarra explains that LINPACK wrote one
version of all the algorithms (in so far as possible) for four
different matrix element types.  In C++ terms, the types are
\texttt{float}, \texttt{double}, \texttt{complex<float>}, and
\texttt{complex<double>}.  LINPACK developers then used string
processing scripts to generate Fortran code for each of the four data
types from this ``abstract'' representation.  LINPACK and its
successor LAPACK use functionality comparable to C++'s
\texttt{numeric\_limits} in order to implement linear algebra
algorithms that are correct and accurate with respect to
floating-point arithmetic, without hard-coding floating-point traits
directly into the algorithm \cite{P1370R0}.

One of the authors, Mark Hoemmen, has been a Trilinos
\cite{heroux2005trilinos} developer since 2010.  One of the most
heavily used parts of Trilinos are the Teuchos BLAS and LAPACK
wrappers.  These are very thin C++ wrappers over the BLAS and LAPACK,
that are templated on the matrix element type (what Trilinos calls the
\texttt{Scalar} type).  The only thing ``C++'' about them, and their
main value, is that they let users write generic C++ code that uses
BLAS and LAPACK functionality.  The wrappers have existed in more or
less their current form for several years longer than the author's
tenure as a Trilinos developer, yet few Trilinos developers complained
and none actually offered a replacement.  This shows that a generic
C++ BLAS and LAPACK binding might satisfy many users, even if it has
no more features than the BLAS and LAPACK themselves.

\subsection{BLAS: Draw the line between libraries
  based on developer expertise}
\label{SS:other-langs:BLAS}

Summary: The BLAS (Basic Linear Algebra Subprograms) set a good
precedent, by drawing the line between libraries based on whether
developers are more likely to be numerical linear algebra experts or
performance tuning experts.

The BLAS (Basic Linear Algebra Subprograms) Technical Forum defines a
standard Fortran 77, Fortran 95, and C interface for linear algebra
operations \cite{BLAS-standard}.  The Standard was published in 2002
and includes dense, banded, and sparse linear algebra operations.
However, the heart of the BLAS is dense and banded linear algebra
operations.  The BLAS in some form has been around since the 1970's
\cite{lawson1979blas,dongarra2005history}, and its dense core has been
in recognizably modern form since 1990 \cite{dongarra1990blas3}.

One benefit of the BLAS is that it draws the line between libraries
based on likely developer expertise.  Efficient BLAS implementation
requires expertise in performance tuning and computer architecture,
but not so much in linear algebra.  Accurate and efficient
implementation of LINPACK and LAPACK calls for understanding numerical
linear algebra and floating-point arithmetic.  The intent was that
computer vendors would optimize the BLAS, and LAPACK would spent
almost all of its time running inside the optimized BLAS \cite[``The
BLAS as the Key to Portability'']{LAPACK-Users-Guide}.  That way,
numerical linear algebra experts could get good performance by
thinking about algorithms at a high level, e.g., by blocking to favor 
operations with more reuse (``BLAS 3'').

This boundary may be a bit fuzzy.  For example, BLAS implementations
can contribute a lot to accuracy in floating-point arithmetic.  Thus
justifies projects like mixed-precision BLAS \cite{lawn149} and
Reproducible BLAS\footnote{See publications list here:
  \url{https://bebop.cs.berkeley.edu/reproblas/}.}, and proofs that
numerical linear algebra algorithms are numerically stable even if
they use Strassen-like implementations of matrix-matrix multiply
\cite{demmel2007fast}.  Furthermore, the set of core linear algebra
operations may evolve over time, as computer architectures and
algorithms evolve.  The version of the BLAS that LINPACK used only
provided vector-vector operations (what later became known as ``BLAS
1'') \cite{lawson1979blas}; this made sense at the time, because those
were the right operations to optimize on vector computers.  As
floating-point operations became much faster than memory operations
\cite{wulf1995hitting} and computer architectures became cache based,
algorithm developers realized that they needed to redesign matrix
algorithms to reuse data better.  (One fruit of that redesign is the
LAPACK project \cite{anderson1990lapack}.)  This motivated them to
extend the original BLAS to support matrix-vector
\cite{dongarra1988blas2} (``BLAS 2'') and matrix-matrix
\cite{dongarra1990blas3} (``BLAS 3'') operations, that have more reuse
than the original vector-vector operations.

Even though the boundary between ``what belongs to BLAS'' and ``what
belongs to LAPACK'' has shifted over time, both library developers and
users appreciate that there is a boundary.

\subsection{High Performance Fortran: Make expensive operations
  explicit}\label{SS:other-langs:HPF}

Summary: If a library (or language) makes time-consuming operations
look no different than fast operations, then users might have trouble
debugging performance issues.

The High Performance Fortran (HPF) programming language
\cite{Kennedy2007,Kennedy2011} inherited Fortran's native support for
multidimensional arrays.  It added support for parallel arithmetic
operations on these multidimensional arrays, that might be distributed
in parallel over multiple processors.  Compare also to the 2-D block
cyclic data distributions that ScaLAPACK \cite{Blackford1997} supports
for dense matrices.  2-D block cyclic distributions include block and
cyclic layouts, both 1-D and 2-D, as special cases.

HPF had a number of issues that hindered its adoption.  One issue
relating to language design, is that the ``the relationship between
what the developer wrote and what the parallel machine executed [was]
somewhat murky'' \cite[p. 13]{Kennedy2007}.  The language did not make
clear what operations could result in expensive parallel
communication.  This made it hard to diagnose suboptimal performance
without dropping down to the parallel equivalent of assembly language
(in this case, generated Fortran + MPI communication calls).

An important lesson learned is that expensive operations should be
explicit.  For example, distributed-memory parallel linear algebra
libraries like Trilinos \cite{heroux2005trilinos} forbid arithmetic
operations between vectors that do not have the same distribution,
unless the user first explicitly invokes a data redistribution
operation on one of the vectors.  Authors of a C++ standard linear
algebra library will need to think about this.  For example, C++
operator overloading is convenient, but it risks hiding computational
expense behind the mask of simple arithmetic.

One of HPF's motivations was to hide the details of distributed-memory
parallel communication.  However, many distributed-memory parallel
libraries and applications wrote to a lower-level programming model
(e.g., MPI), but built higher-level library abstractions around it.
For example, discretizations of partial differential equations that
use finite-element or finite-volume method have a ``boundary
exchange'' or ``ghost region exchange'' primitive, that communicates
whatever data a parallel process needs in order to compute the next
time step or nonlinear solve step.  Good library design hides almost
all parallel communication behind a small number of library
primitives, at least for applications that have regular communication
patterns.

\section{Other standardization efforts}
\label{S:other-standards}

Here, we briefly mention related linear algebra standardization
efforts, the GraphBLAS (Section \ref{SS:other-standards:GraphBLAS})
and Batched BLAS (Section \ref{SS:other-standards:batched}).  Should
we consider those in scope for a standard C++ linear algebra library?
We should at least know about them and make a conscious decision to
include or exclude them.

\subsection{GraphBLAS}
\label{SS:other-standards:GraphBLAS}

Summary: The GraphBLAS offers to make a generic sparse linear algebra
library useful for implementing graph algorithms.  It requires custom
matrix element types and custom arithmetic, that could impose
interesting requirements on a linear algebra library.

The GraphBLAS Forum is a recent, ongoing effort to standardize
``building blocks for graph algorithms in the language of linear
algebra.''\footnote{\url{http://graphblas.org/index.php?title=Graph_BLAS_Forum};
  see also the brief position paper \cite{mattson2013standards}.}  The
point of the GraphBLAS is that it's both possible and performant to
express many graph algorithms using a small number of linear algebra
primitives, such as sparse matrix-matrix multiply.  One can think of
the GraphBLAS as a subset of traditional sparse BLAS functionality,
but with a few ``wrinkles'' that differ from what a BLAS implementer
might expect.  For example:
\begin{itemize}
\item Vectors are ``sparse by default''; for example, one can express
  one step of breadth-first search by a sparse matrix-vector multiply,
  but this is only efficient if the vector is stored sparsely.
\item Users need to be able to define custom arithmetic types, and
  custom arithmetic operations to replace plus, times, and zero (the
  arithmetic identity and multiplicative annihilator).  Desired
  arithmetic types may be real numbers (with different operations),
  Boolean, or subsets of a set of integers \cite{kepner2016math}.  The
  latter is interesting because na\"ive implementations might store
  subsets in a way that requires dynamic allocation.
\end{itemize}

Many C++ linear algebra libraries permit custom arithmetic types and
operations.  We argue in Section \ref{SS:other-langs:LINPACK} that
genericity on the matrix and vector element types is one of the most
important features that a C++ linear algebra library can offer.
However, how ``generic'' is ``generic''?  The Eigen library assumes
that the usual arithmetic operations (including division) work, and
imposes some other interface requirements.\footnote{See
  \url{http://eigen.tuxfamily.org/dox/TopicCustomizing_CustomScalar.html}.}
Other libraries that claim to be generic have similar requirements.

One of the authors, Mark Hoemmen, has some experience with this in the
Tpetra linear algebra library \cite{baker2012tpetra} in the Trilinos
project \cite{heroux2005trilinos}.  Tpetra's vector and matrix data
structures are templated on the element type (\texttt{Scalar}), but
Tpetra is more restrictive about the set of types it permits than
other libraries might be.  While this is may be a design flaw in
Tpetra, it's likely that other existing C++ linear algebra libraries
have this restriction as well.  Thus, understanding the reasons for
the restriction could help expose pitfalls when attempting to use an
existing library to implement the GraphBLAS.  Here are some reasons
why Tpetra does not permit arbitrary matrix entry types:
\begin{itemize}
\item Tpetra needs the types to work in CUDA code.  This excludes
  types that do dynamic memory allocation inside, such as the ``na\"ve
  implementation of subsets of a set of integers'' type mentioned
  above.  A small number of Trilinos developers go through a lot of
  trouble to make Tpetra work for a few types that have run-time
  sizes, where all instances of the type have the same run-time size.
\item Tpetra has some coupling with downstream preconditioners and
  iterative solvers.  These generally assume that the type of matrix
  and vector entries implements (mathematical) field operations like
  division.  The ``custom arithmetic'' mentioned above is more like a
  mathematical semiring; for example, it only requires plus and times.
  Tpetra does permit specializations of vectors and matrices to have a
  different ``dot product result type'' than the matrix / vector entry
  type; this could be a path forward for refactoring.
\item Tpetra needs to be able to communicate instances of the matrix
  entry type using MPI (the Message Passing Interface for
  distributed-memory parallel programming).  Permitting types that
  have different run-time sizes would likely require more
  communication (more volume or more rounds or both).  Tpetra's
  communication layer would become more complicated, since Tpetra
  would need to optimize for the common case of built-in types like
  \texttt{double}.
\end{itemize}

The GraphBLAS does have a reference C++ implementation.\footnote{See
  \url{https://github.com/cmu-sei/gbtl}.}  However, any
parallelization would need to think carefully about the implications
of custom arithmetic.  These might include load balancing (for types
with different run-time sizes), thread safety and scalability of the
type, and whether the type could work as a reduction result.  (The
OpenMP standard only recently started allowing custom types for
reduction results, for example.)

\subsection{Batched BLAS}
\label{SS:other-standards:batched}

Summary: Batched BLAS is useful, but has many more interface options.
A data structure like \texttt{mdspan} \cite{P0009r8} might be useful
for exposing interfaces generic on layout.

We mentioned ``batched'' linear algebra interfaces briefly above.
These perform many small dense operations in an optimized way,
generally with the intention of parallelization
\cite{dongarra2016batched}.  Different vendors (e.g.,
Intel\footnote{See
  \url{https://software.intel.com/en-us/articles/introducing-batch-gemm-operations}.}
and NVIDIA\footnote{See
  \url{https://docs.nvidia.com/cuda/cublas/index.html}.}) have
different interfaces, and the interface matters a lot for performance.
Relton et al.\ survey different interface options
\cite{relton2016comparison}.

The \texttt{mdspan} \cite{P0009r8} multidimensional array data
structure could give users a way to store a batch of small dense
matrices in an optimized layout.  It also could give a linear algebra
library a way to express interface polymorphism.

\section{Conclusions}
\label{S:conclusions}

The authors offer here some subjective views on the goals of a C++
linear algebra standardization effort.

A standard C++ linear algebra library will likely have many users who
are not practitioners of traditional ``scientific and engineering
computing.''  They might be interested in computer graphics, gaming,
computer vision, machine learning, or graph algorithms.  Some users
care about matrices and vectors, others about tensors, and still
others about quaternions and octernions.  Nevertheless, we think all
these fields have enough in common that a unified effort can pay off.

A linear algebra library will want to expose iteration over
multidimensional sequences or index spaces, much like the C++ Standard
Template Library or Ranges do with single-dimensional sequences.
Einstein index notation for tensors is a general way of expressing
this, but is not the only way.  Users want to access the entries of
vectors, matrices, and tensors, so we will need to come up with good
abstractions for this.

Users want code to look as much like mathematical notation as
possible.  This can have a performance cost, but libraries can reduce
or eliminate that cost using techniques like expression templates.
Nevertheless, developers must take care to avoid pitfalls like
incorrect use of \texttt{auto} with temporary expressions, and long
build times.  Users also need to take responsibility for compilation
time and memory issues by breaking up long compilation units.

One of the most important lessons learned from the BLAS, LINPACK, and
LAPACK, is that it pays to draw the boundaries between libraries based
on the likely expertise of the developers who would work on each
library.  For example, implementing ``inverse of a matrix'' accurately
requires much more numerical linear algebra knowledge than
implementing ``dot product of two vectors.''  Writing a fast dense
matrix-matrix multiply takes much more knowledge about computer
architecture than writing a reasonably efficient dense LU
factorization.  C++ does put domain-specific knowledge into its
Standard Library; the calendar and time zone library \cite{P0355r7}
that was voted into C++20 is a good example.  Nevertheless, C++
Standard Library developers, despite their fantastic talents,
generally are not experts in numerical linear algebra.  This suggests
that a C++ linear algebra library should carefully separate the
operations whose implementation calls for numerical linear algebra
expertise, from the more ``computer science-y'' operations.

Examining the history of the BLAS Standard, and related
standardization efforts like the GraphBLAS and Batched BLAS, can help
us more consciously define the scope of a standard C++ linear algebra
library.  Goals like ``generic on matrix element type'' or
``zero-overhead abstraction'' have concrete implications on interface
and implementation complexity, for example.

We find it encouraging that C++ is seriously thinking about
standardizing a linear algebra library, and hope that this essay
contributes to that effort.


