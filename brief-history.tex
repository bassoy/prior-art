\chapter{Brief History of Linear Algebra Libraries in C++}

\section{Object-Oriented Numerics in the 1990s}

Linear algebra libraries in C++ at first evolved from the general category of
``object-oriented programming.'' For example, the First annual Object-Oriented
Numerics Conference (OON-SKI) took place in
1993.\footnote{\url{https://www.hindawi.com/journals/sp/si/250702/}}
Rogue Wave, a C++ compiler company, sponsored OON-SKI,
so the conference logically focuses on work in C++.
``Numerics'' here means ``numerical computation'' or ``scientific computing'':
computations on large arrays of floating-point numbers, doing things like
discretizing differential equations or statistical data analysis.\footnote{
Authors use adjectives like ``numerical'' to describe scientific and engineering
computation.}  The introduction to the corresponding journal special issue
described the phenomenon of ``object-oriented numerics'' as follows:

\begin{quote}
  \ldots [W]e are observing the emergence of a subdiscipline: the use of
  object-oriented techniques in numerics. But what we are really seeing is
  something even more profound: finally rejoining of scientific computing with
  the science of computers. Traditionally, programming has been done by
  engineers, physicists and mathematicians with little or no training in
  computer science. Now, however, we are seeing an infusion of ideas coming from
  computer science world into the scientific computing world, brining along
  modern ideas on how to structure complex numerical code. Object-oriented
  techniques is merely one of many such ideas \cite{Vermeulen1993}.
\end{quote}

The introduction talks about the issues like needing to teach compilers how to
fuse loops and avoid temporaries when doing overloaded-operator arithmetic on
arrays in C++. It also shows the existence of C++ libraries for a variety of
applications, including a library of multidimensional arrays, and the
integration of C++ with distributed-memory parallel computation.

One can see the explosion of interest in ``object-oriented numerics'' by the large
variety of conferences that sprang up in the mid-1990's.\footnote{See e.g.,
\url{http://www.math.unipd.it/~michela/OP.htm#conferences}.} Interest died
down later, but this perhaps reflects the stage in mathematical software development
where users came to accept externally developed libraries as part of their
applications. In the United States, this may correspond to the
Department of Energy's Accelerated Strategic Computing Initiative
(ASCI) program, which was founded in 1996.\footnote{\cite{messina1996asci}}
An important part of ASCI was ``[s]oftware design and development''
for complicated ``multi-physics scientific applications.''
Before ASCI, experience with such software development was ``limited to a few
isolated projects.'' ASCI aimed to ``carry out multiple substantial simulation
projects that will provide proofs of principle, at scale, of software development
methodologies and frameworks that provide portability, extensibility, modularity,
and maintainability while delivering acceptable performance.''

Another contribution to the development of C++ linear algebra
libraries may have been the ``killer micros'' revolution of the early
1990's.  Killer micros were consumer-grade, low-cost microprocessors
whose performance threatened and eventually overtook that of expensive
vector supercomputers.\footnote{\cite{killermicros1991}} This
revolution disrupted Fortran's dominance and let developers write more
complicated codes.  Vector supercomputers like the Cray 1 depended on
custom vectorizing Fortran compilers for good performance.  ``Scalar''
code ran much more slowly on the Cray 1 than vectorized code, and
getting code to vectorize took careful dialog with the compiler.
Killer micros had less optimized hardware, but also had performance
less sensitive to software design.  Hardware performance improvements
tracked the 18-month Moore's Law curve, so users could just wait for
the next processor generation, instead of expensively optimizing their
code.  This gave developers more freedom to use languages other than
Fortran, and to write more complicated codes that depended less on
vectorization.  Simultaneously, new distributed-memory parallel
computers emerged, along with new languages to program them.  For
example, the technical report describing Connection Machine's CM-1
only mentions one programming language: *Lisp, an ANSI Common Lisp
derivative.\footnote{\cite{kahle1989cm1}} Applications running on CM-2
also used C*, a parallel superset of ANSI C.  Other distributed-memory
parallel computers of the era favored C and offered C++ compilers.
Library-oriented approaches to parallel computing, like PVM and MPI,
came with C as well as Fortran bindings.  The switch from Fortran and
the increasing complexity of software made C++ and richer library
abstractions more attractive.

\section{Templates}

C++ had a reputation for poor performance compared with Fortran. Even developers
willing to write C++ in ``numerical'' codes considered it better to use C++ has a
high-level coordination language, and reserve lower-level languages like C for
tight loops.\footnote{\cite{Arge1997}} Developers saw C++ templates, in particular
expression templates, as an optimization technique that could close the
performance gap. Expression templates would let developers write compact,
abstract code that ``looks like math,'' yet optimizes by fusing loops and avoiding
temporaries. For example, the Dr.\ Dobbs article\footnote{\cite{dobbsblitz1997}}
on the Blitz++ library,\footnote{\cite{blitz2005}} written by the
library's author, focuses on expression templates for vector operations.

Developers also recognized the cost of virtual method calls in C++,
especially in inner tight loops, and used templates to reduce the cost
of run-time polymorphism.  For example, the Bernoulli Generic Matrix
Library uses the ``Barton-Nackman trick,''\footnote{\cite{Barton1994}}
a special case of the ``Curiously Recurring Template Pattern,'' to
turn run-time polymorphism into compile-time polymorphism.\footnote{In
  \cite{Mateev2000}, authors cite \cite{Veldhuizen2000}}

Early libraries that relied on templates suffered due to incomplete compiler
implementations. For example, Blitz++'s installation process exercises the
compiler to test language feature compliance. Its User's Guide recommends that
if the compiler ``doesn't have member templates and enum computations, just give
up.''\footnote{See Section 1.4.3 of \cite{blitz2005}.} A comparable library,
POOMA (Parallel Object-Oriented Methods and
Applications),\footnote{\url{http://www.nongnu.org/freepooma/tutorial/introduction.html}}
pushed the boundaries of what the available C++
compilers could handle. Chris Luchini, a POOMA developer, recalls that the
project exposed many compiler bugs.\footnote{\cite{hoemmen2018history}}
Many compilers lagged behind the C++ standard,
only implemented a subset of features,
and generated slow code.\footnote{\cite{Mateev2000}}

Software for scientific computing may need to build with several different
compilers and run on different kinds of hardware. Lack of consistently complete
implementations of templates challenged portability requirements and restricted
adoption. For example, in the Trilinos software project, a requirement to
support a C++ compiler with incomplete template support drove the project to
forbid templates in its foundational linear algebra library, Epetra.\footnote{\cite{hoemmen2018history}}

\section{POOMA}

The POOMA (Parallel Object-Oriented Methods and Applications) project
was most active 1998-2000. POOMA's goal was to support structured grid
and dense array computations. As per oral history\footnote{See
  \cite{hoemmen2018history}.} and POOMA's documentation, the team had
a particular interest in SGI Origin shared-memory parallel
computers. POOMA shares features with more recently linear algebra
libraries, such as polymorphism on storage layout and parallel
programming model, so it is worth studying for historical lessons.

POOMA's main data structure is Array. Array has three template parameters:
the rank (the number of dimensions),
the entry type (e.g., \texttt{double}),
and the ``engine.''
Engines are about storage of data.
They correspond somewhat to the Accessor policy
in the \texttt{mdspan} multidimensional array proposal.\footnote{\cite{P0009r8}}
An engine implements access to entries of the Array.
Entries could actually exist in some storage somewhere,
or they could be computed from indices and not actually stored. Engines also
describe parallel distribution somewhat -- e.g., through the MultiPatch engine.\footnote{\url{http://www.nongnu.org/freepooma/tutorial/tut-04.html}}

POOMA's Internals and Ranges let users construct possibly strided
multidimensional index ranges. These features let users write very general
indexed loops, like in the ZPL programming language.\footnote{\cite{Chamberlain1998}}
However, POOMA users had to work a bit harder on distributed-memory parallel
systems, to expose ``guard regions'' with redundant storage on process boundaries.\footnote{This essentially means that POOMA did not do implicit boundary
  exchange.  High-performance computing experts like to expose and reify the
  parallel distribution, and any redistribution operations. This helps them
  avoid communication and data movement, and makes parallel synchronization
  semantics clear.  This is also a bit of a reaction to High-Performance
  Fortran, where even copying from one array to another could require parallel
  synchronization.
}

The POOMA project had to ``discover'' experimentally how C++ templates work, and
develop their own idioms. For example, the developers learned that C++
does not permit templating on return type and then deducing the return type.\footnote{\url{http://www.nongnu.org/freepooma/tutorial/tut-03.html}}
As mentioned above, POOMA developers also had to explore the limits of compiler
correctness and performance.

\section{C++ for More Radical Optimizations}

As experience with C++ templates increased, some developers applied them to more
radical code optimizations.  For example, the Bernoulli Generic Matrix Library
used templates to generate optimized sparse matrix codes from a high-level
specification.\footnote{\cite{Ahmed2000} By the time, I (Mark Hoemmen)
encountered the Bernoulli project, it had abandoned C++ code generation in
favor of OCaml (or some other ML derivative)-based code generation framework.
My guess is that avoiding intermediate high-level C++ step improved run-time
performance and avoided compiler correctness issues.}  Bernoulli used a kind of
relational algebra (described in detail in the PhD dissertation) that is
somewhat analogous to the C++ Ranges proposal \cite{Niebler2018}, in that it gives
users a general way to describe operations over sequences, while optimizing by
avoiding storage of temporary intermediate sequences.

\section{Lessons Learned from Efforts in Other Programming Languages}

\subsection{LINPACK}

Dongarra\footnote{\cite{dongarra2005history}}
gives an oral history of standardization of popular Fortran linear algebra
libraries, including EISPACK and LINPACK.  LINPACK came later.  Here is a longer
quote from this oral history explaining LINPACK's choice to rely on the BLAS:

\begin{quote}
  Since linear systems have perhaps a broader impact, LINPACK was
  going to a wider audience, and we felt that it would have a larger acceptance.
  This package was designed at a time when the biggest computers available were
  the vector computers. The vector supercomputers were just coming onto the
  scene, and the package was designed with vector computers in mind, so the
  package was designed to rely on an underlying set of routines called the BLAS
  (the Basic Linear Algebra Subprograms). The BLAS are a set of kernels which
  form the computational core of LINPACK; they are the vector operations that
  are going to be done over and over again in the package. The BLAS were a set
  of standard routines which were formed right before LINPACK was really kicked
  off, and we made a decision to use them.
\end{quote}

In so far as possible, the project wrote one version of the algorithms for four
different data types (two different real precisions and two different complex
precisions), and generated Fortran code for each of the four data types from
this ``abstract'' representation.

\subsection{High Performance Fortran}

The High Performance Fortran (HPF) programming
language\footnote{\cite{Kennedy2007,Kennedy2011}} inherited Fortran's
native support for multidimensional arrays.  It added support for
parallel arithmetic operations on these multidimensional arrays, that
might be distributed in parallel over multiple processors.  Compare
also to the 2-D block cyclic data distributions that
ScaLAPACK\footnote{\cite{Blackford1997}} supports for dense matrices.
2-D block cyclic distributions include block and cyclic layouts, both
1-D and 2-D, as special cases.

\section{Vector Spaces and Parallel Data Distributions}

How can I tell if I'm allowed to add two vectors together, or multiply two
matrices?  Is it enough for their dimensions to be compatible?  Mathematicians
would point out that two vector spaces might still differ, even if they have the
same dimension or are otherwise isomorphic.  I can't add a coordinate in 3-D
Euclidean space to a quadratic polynomial with real coefficients, just like I
can't add meters to seconds.  Like a physical unit, a vector space is a kind of
``metadata.''  Equating all isomorphic vector spaces strips off their metadata.

A distributed-memory parallel data distribution is very much like a vector
space. It takes an $N$-dimensional vector space and imposes a two-dimensional
index $(p,I(p))$ on it, Here $p \in [0,P)$ is the parallel process index (the
``rank,'' in MPI terms), and $I(p)$ is the set of indices that live on Process
$p$. The only thing that would make this a finite-dimensional vector space is
the field over which the entries of a vector are defined. Even if two vectors
$x$ and $y$ have the same dimension $N$, if the two vectors have different
parallel distributions, I can't add them together without communication.
``Communication'' here may mean different things on different parallel computers,
but this generally means some combination of moving data between processors
(either through a memory hierarchy, or across a network), and synchronization
between processors. Communication is expensive relative to floating-point
arithmetic.\footnote{\cite{wulf1995hitting,Blackford1997}}
It also affects correctness,
because it may introduce deadlock,
depending on what surrounding code does.
Thus, programmers like to see communication made explicit,
even if it is hidden behind a convenient interface.

Linear algebra libraries can make this easier for programmers through the
abstract language of vector spaces.  For example, the library can let users
construct and pass around a parallel distribution using the two-dimensional
indexing structure $(p, I(p))$ mentioned above.  Users then create
matrices and vectors using distribution objects created in this way.  The
library can forbid implicit arithmetic operations between different vector
spaces, but can make data redistribution and/or ``communicating'' arithmetic
operations explicit.  Users can also get the data distributions out of a matrix
or vector, and check themselves whether two different distributions are the
same.  It's easier to explain all this to users in the abstract language of
vector spaces.

Even if I don't care about distributed-memory parallelism, I may still care
about shared-memory parallelism (threads) and memory affinity.

\section{A Matrix has Four Vector Spaces}

Matrices fulfill two different roles.  First, matrices are 2-D data containers.
Their rows have a data distribution, and their columns do also.  Second, I can
do matrix-vector products with a matrix.  This makes a matrix a function from
its domain vector space to its range vector space.
